\documentclass[]{article}

% \usepackage{lmodern}
% \usepackage{amssymb,amsmath}
% \usepackage[T1]{fontenc}
\usepackage[utf8]{inputenc}
\usepackage{microtype}
\usepackage{longtable,booktabs}
\usepackage[unicode=true]{hyperref}
\usepackage{graphicx}
\usepackage{lscape}
\usepackage{titlepage}
\usepackage{listings}
\usepackage{tabularx}
\usepackage{todonotes}


\lstset{
captionpos=b
}
\begin{document}
\maketitlepage
\section{Ontology}
\subsection{Organization}
Sander made a slack-channel where everyone of the class was invited to. On this channel, we discussed how we would make the ontology. 

The first idea was to create a global ontology around calories. This was the one thing that both the food groups and the exercise groups had in common. Because there was not anything else, we left it that way and suggested to start working in separate groups. 
We met two times on the slack-channel, where we did not agree on anything more than the definition of calorie. Therefore we suggested to meet face to face on a day where we did not had any classes. 

Every group had to prepare the definitions they would suggest to use and search for entities in their database. We had this meeting in the morning, with all groups but one, group number 8. This group was the only group that did not reply in the slack-channel, the Facebook chat or on the forum on PointCarré. With all other groups present we made the first ontology where every group could map to it and agreed to it. When almost finishing the ontology, someone noticed that calorie is not a nutritional value. In the ontology we had made calorie a nutritional value. We had a discussion whether we should change it and put calorie in the ingredient entity or leave it that way. Because our definition of nutritional values was also corresponding to calories, we left it the way we first agreed to it because every group could map onto this ontology. 

For this meeting, everyone worked together, one person typed in a document and one person drew on the blackboard. We agreed to finish the ontology, by putting it into Web Protégé, on the same day in the afternoon. During the second meeting face to face, we will look for other commonalities between different groups and maybe develop other ontologies.
\subsection{Other discussions}
We discussed what other groups had in common with our proposal. Because there where no things that where worthwhile exchanging we did not add an extra ontology.

We had the discussion with all of the groups whether or not we should exchange information about our users or the accounts. We did not see any benefits and for privacy reasons for the user we decided not to exchange user information and therefore not include this in the ontology.  
\section{Glossary}
\textit{Recipe} \\
A collection of ingredients and /  or the steps necessary to combine the ingredients into some food. It contains a list of ingredients, an id and optional steps. \\ \\
\textit{Food} \\
Any substance that can be eaten or drunk. When consumed, it provides a nutrition. Food can be an ingredient or the result of a recipe. It only has a name. \\ \\
\textit{Ingredient} \\
An ingredient is a substance containing nutrients; an ingredient can be a food. It contains an optional brand, a name and a list of nutrients.  \\ \\
\textit{IngredientOfRecipe} \\
The link between a recipe and an ingredient. It depicts the amount of some ingredient necessary to complete a recipe. It contains of an ingredient, a recipe, the amount needed and the unit for the amount. This can be litre or gram or something else.\\ \\
\textit{Nutrient} \\
A substance that provides nourishment essential for the maintenance of life and for growth. It can be fat, sugar, carbohydrate, protein, sugar or calories. It has a nutrient\_type and a unit.  \\ \\
\textit{Ingredient attribute} \\
The link between a nutrition and an ingredient. It depicts the amount of some nutrient per 100 gram of some ingredient. It has an ingredient and a nutrient.\\ \\
\end{document}
