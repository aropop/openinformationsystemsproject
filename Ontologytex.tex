\documentclass[]{article}

% \usepackage{lmodern}
% \usepackage{amssymb,amsmath}
% \usepackage[T1]{fontenc}
\usepackage[utf8]{inputenc}
\usepackage{microtype}
\usepackage{longtable,booktabs}
\usepackage[unicode=true]{hyperref}
\usepackage{graphicx}
\usepackage{lscape}
\usepackage{titlepage}
\usepackage{listings}
\usepackage{tabularx}




\begin{document}
\maketitlepage
%\section{Ontology}

\abstract{
We are building an application which allows the user to keep track of the amount of calories that he/she eats a day.  The application provides a way to track the desired amount of food the user should eat in order to stay on the same weight. Based on the amount of calories the user needs, some recipes are proposed. The user could provide a desired weight, which will lead to an adaptation of the needed calories intake.}

\section{Introduction}
This report elaborates on the process of creating the ontology for the course Open Information Systems. It describes the organization within our team and how the collaboration with the other teams took place. It shows who contributed to the ontology, which aspects went smoothly and which parts of the ontology were points of discussion.  




%\newpage

\section{Ontology}

\subsection{Organization}

In order to be able to communicate easily between the different groups we decided, after some initial contact on Facebook, to create a slack channel. This channel was created by Sander Lenearts and at least one person from each group was contacted so all the groups and all their members could join. This slack channel has as an advantage that it was possible to start a first discussion during the vacation without having to have everybody physically present at the same place. 
\newline
\newline
\noindent
In the first meeting that took place on slack we wanted to create a global ontology. The only thing that both the food groups and the exercises groups had in common was calories. We decided on a definition that both had a relationship to food and to exercises without getting too technical. Another point of discussion was the unit. Some people wanted to have the SI unit but since the calories are more widely used it was decided to use calories as unit. Since there were no more commonalities, we decided to work with the food group separately. 
\newline
\newline
\noindent
The second slack meeting, for the food team, there was at least one member of every team besides team 8. There we started with a definition for an Ingredient. A huge point of discussion was whether an ingredient had a brand. Group 5 really wanted this incorporated while this was the only group that needed this aspect. It was proposed that the brand could be part of a more specialized ontology but group 5 didn't agree with this so it was decided that brands would be part of it but clearly stating that they were optional. Since slack has as drawback that it doesn't keep the complete channel history and since discussions on slack can be rather uncoordinated we decided to have the next meeting face to face. 
\newline
\newline
\noindent
Every group had to prepare a glossary and search for entities in their database. During this meeting, that took place Monday 11 April during the morning, all groups but one, group number 8, were represented by at least one member. Group 8 did not reply in the slack-channel, the Facebook chat or on the forum on PointCarré. We discussed other ways of contacting them, and decided we would personally contact them during another class. During this meeting we discussed the first food ontology. There was a discussion about whether only to incorporate calories or if other nutritious values were necessary as well. In the end we decided to go with the definition that group 7 proposed since every group was able to map on this definition and since this was the design that was general enough so that the ontology could be easier to reuse. The incorporation of recipe, food and ingredient, and their relation was one of the easiest to decide on. Persons aren't included in our ontology since we didn't think this was information that would be shared and we had some concerns about privacy. Furthermore we discussed that everybody had some kind of concept to describe either ingredients or recipes, tags in our case. But since the definition of this was so different for each group we settled on not including them at this point in our ontology and we noted that we can always come back to it. During this meeting the focus however did seem to be a lot on how our databases were designed and sometimes group 5 did seem to focus too much on their specific implementation. After agreeing on most definitions and almost finishing the basic ontology, someone noticed that calories is not a nutritional value. We had a discussion whether we should change it and put calorie in the ingredient entity or leave it that way. Where group 5 proposed to put calories in ingredients and have an extra energy object(is dat een klasse of een object?) that would enable us to use the energetic value of an ingredient in joule or in kcal. Because calories did fit our personal definition of nutritional values, we left it the way we first agreed to it because every group could map onto this ontology. The collaboration went good and we had one person typing our glossary in a Google doc while somebody else drew the different entities on the blackboard. 
\newline
\newline
\noindent
During the evening meeting of Monday 11 April, every group was present, besides group 8. During this meeting we had the intend to put the ontology in web Protégé. We encountered some difficulties with the platform and decided to ask for more information and get together another day. 
\newline
\newline
\noindent
We tried having another meeting on Wednesday 13 April where group 2 and group 1 tried putting the ontology in Web Protégé. Since we still encountered problems we decided to send another mail for more information and ask extra information during the next class of Open Information Systems. 
\newline
\newline
\noindent
During the meeting of Monday 18 April, after having asked questions during the classes that day, all groups were present. We put the ontology in web protege. Furthermore we had an extra discussion since we didn't want to have calories in the nutritional value since we didn't like the fact that calories weren't in fact nutritional values. Group 5 also brought up that they didn't like the relationship with amount/nutritional value since it was too complex. They proposed (again) to have an extra attribute (?? juiste benaming??) for energy and have calories directly linked to ingredients. After hearing the arguments it was decided to change the name, since calories aren't nutritional values. But to keep the relationship with amount(changed name to quantity since this is more informative)/nutritional value since this one is more general. We like to note that while group 8 was present they didn't have any input. 

\subsection{Other discussions}
We discussed what other groups had in common with our proposal. Because there where no things that where worthwhile exchanging we did not add an extra ontology.

We had the discussion with all of the groups whether or not we should exchange information about our users or the accounts. We did not see any benefits and for privacy reasons for the user we decided not to exchange user information and therefore not include this in the ontology.  
\section{Glossary}
\textit{Recipe} \\
A collection of ingredients and /  or the steps necessary to combine the ingredients into some food. It contains a list of ingredients, an id and optional steps. \\ \\
\textit{Food} \\
Any substance that can be eaten or drunk. When consumed, it provides a nutrition. Food can be an ingredient or the result of a recipe. It only has a name. \\ \\
\textit{Ingredient} \\
An ingredient is a substance containing nutrients; an ingredient can be a food. It contains an optional brand, a name and a list of nutrients.  \\ \\
\textit{IngredientOfRecipe} \\
The link between a recipe and an ingredient. It depicts the amount of some ingredient necessary to complete a recipe. It contains of an ingredient, a recipe, the amount needed and the unit for the amount. This can be liter or gram or something else.\\ \\
\textit{Nutrient} \\
A substance that provides nourishment essential for the maintenance of life and for growth. It can be fat, sugar, carbohydrate, protein, sugar or calories. It has a nutrient\_type and a unit.  \\ \\
\textit{Ingredient attribute} \\
The link between a nutrition and an ingredient. It depicts the amount of some nutrient per 100 gram of some ingredient. It has an ingredient and a nutrient.\\ \\
\end{document}
